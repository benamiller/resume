\documentclass[10pt]{article}
\usepackage[margin=0.65in]{geometry}
\usepackage{hyperref}
\usepackage{xcolor}
\usepackage{titlesec}
\usepackage{enumitem}
\definecolor{linkblue}{HTML}{0366d6}
\hypersetup{
    colorlinks=true,
    urlcolor=linkblue
}
\titleformat{\section}{\large\bfseries}{}{0em}{}
\setlength{\parindent}{0pt}

\begin{document}

\begin{center}
    {\Huge\textbf{Benjamin Miller}}\\
    \vspace{1mm}
    \href{mailto:EMAIL}{EMAIL} $\bullet$ PHONE $\bullet$
    \href{https://github.com/benamiller}{github.com/benamiller}
\end{center}

\section*{Education}
\textbf{MSc, Computer Science (AI/ML)} \hfill \textit{Jan 2025 -- May 2026 (Expected)}\\
University of Illinois Urbana-Champaign \hfill 4.0 GPA\\
\textbf{BSc, Computer Science (Software Systems)} \hfill \textit{Sep 2019 -- May 2022}\\
University of Victoria \hfill Dean's Entrance Scholarship, President’s Scholarship

\section*{Experience}
\textbf{Backend Java \& DevOps Developer} \hfill \textit{Canadian Tire, Feb -- Nov 2024}
\begin{itemize}[leftmargin=0.25in, itemsep=0pt]
    \item Optimized Azure Databricks pipelines, reducing cloud overhead and resource costs.
    \item Developed geo-replicated Pulsar clusters on Azure AKS, automating disaster recovery.
    \item Automated Kubernetes key rotation, cutting manual intervention time by 75\%.
    \item Resolved critical Pulsar outage, enabling timely quarterly deliverables.
\end{itemize}

\textbf{QA Specialist} \hfill \textit{Redbrick, May 2022 -- Feb 2024}
\begin{itemize}[leftmargin=0.25in, itemsep=0pt]
    \item Engineered automation suites (Playwright, TypeScript), cutting manual QA workload by 50\%.
    \item Improved product reliability by identifying and fixing critical JavaScript bugs.
    \item Facilitated Agile workflows and streamlined code reviews.
\end{itemize}

\section*{Technical Skills}
\textbf{Languages:} Java, Python, TypeScript, JavaScript, SQL, Bash\\
\textbf{Cloud/DevOps:} Azure (Databricks, AKS, Service Bus, DevOps), Terraform, Kubernetes, Helm, Pulsar\\
\textbf{ML Frameworks:} TensorFlow, PyTorch, scikit-learn\\
\textbf{Web:} Spring Boot, React, Node.js\\
\textbf{Databases:} MongoDB, CosmosDB, Azure SQL, PostgreSQL

\section*{Certifications}
\textbf{Deep Learning Specialization} (Coursera, Andrew Ng) \hfill \href{https://coursera.org/share/f5fa4c831a360de7841411165ebabcc4}{Certificate}, \textit{Sep 2024}
\begin{itemize}[leftmargin=0.25in, itemsep=0pt]
    \item Built CNNs for CIFAR-10; developed deep RL models for OpenAI Gym environments.
\end{itemize}
\textbf{Machine Learning Specialization} (Coursera, Andrew Ng) \hfill \href{https://coursera.org/share/c8ccd34bd3358236ad3f17c5907bdf17}{Certificate}, \textit{Feb 2024}
\begin{itemize}[leftmargin=0.25in, itemsep=0pt]
    \item Developed regression/classification models and ensemble methods in scikit-learn.
\end{itemize}

\section*{Projects}
\textbf{mlzero} \hfill \href{https://mlzero.com}{mlzero.com}
\begin{itemize}[leftmargin=0.25in, itemsep=0pt]
    \item Created an interactive ML learning platform with comprehensive public lessons.
\end{itemize}
\textbf{CUDA CNN from Scratch} \hfill \href{https://cnn.c}{cnn.c}
\begin{itemize}[leftmargin=0.25in, itemsep=0pt]
    \item Implementing optimized convolutional neural networks using CUDA and C.
\end{itemize}

\section*{Research Interests}
Deep Learning, Distributed Systems, Reinforcement Learning, Digital Intelligence

\end{document}

